% !TEX encoding = UTF-8
% !TEX TS-program = pdflatex
% !TEX root = ../tesi.tex

%**************************************************************
\chapter{Licenze software}
\label{cap:licenze-software}
%**************************************************************

\intro{Il capitolo in questione effettua una panoramica sulle licenze per il software libero, analizzando quelle utilizzate dalle librerie terze necessarie all'implementazione del software Biztortion e quindi quella utilizzata per il rilascio del suddetto software}\\

%**************************************************************
\section{Software libero}

Il software \hyperref[cap:biztortion]{Biztortion}, corrispondente a parte del lavoro effettuato per la tesi di laurea, è stato sviluppato con l'obiettivo di approfondire il fenomeno della distorsione del segnale audio e di aumentare il mio bagaglio di esperienza da programmatore. Essendo inoltre la tesi di laurea un lavoro di studio e ricerca, risuta logicamente ed eticamente corretto il rilascio e la diffusione del software in questione come \textit{software libero}. \\ \\
Il “Software libero”\footcite{site:software-libero} rispetta la libertà degli utenti e la comunità. In breve, significa che gli utenti hanno la libertà di eseguire, copiare, distribuire, studiare, modificare e migliorare il software. Tramite queste libertà gli utenti (individualmente o nel loro complesso) controllano il programma e le sue funzioni. Quando non sono gli utenti a controllare il programma, allora il programma (che in quel caso è denominato "non libero" o "proprietario") controlla gli utenti. \\
Un programma è software libero se gli utenti del programma godono delle quattro libertà fondamentali:
\begin{enumerate}
    \item Libertà di \textbf{eseguire} il programma come si desidera, per qualsiasi scopo;
    \item Libertà di \textbf{studiare} come funziona il programma e di \textbf{modificarlo} in modo da adattarlo alle proprie necessità. L'accesso al codice sorgente ne è un prerequisito;
    \item Libertà di \textbf{ridistribuire} copie in modo da aiutare gli altri;
    \item Libertà di \textbf{migliorare} il programma e distribuirne pubblicamente i miglioramenti apportati (e le versioni modificate in genere), in modo tale che tutta la comunità ne tragga beneficio. Anche qui l'accesso al codice sorgente ne è un prerequisito.
\end{enumerate}
E' necessario inoltre evidenziare il fatto che software libero non vuol dire non commerciale. Al contrario, con un programma libero deve essere possibile anche l'\textbf{uso commerciale}, lo sviluppo commerciale, e la distribuzione commerciale. Questa politica è di importanza fondamentale, infatti senza questa il software libero non potrebbe raggiungere i suoi obiettivi. \\
Infine certi tipi di regole sul come distribuire il software libero sono accettabili quando non entrano in conflitto con le libertà principali. Per esempio, il \textbf{copyleft}, noto anche impropriamente come "permesso d'autore", è in poche parole la regola per cui, quando il programma è ridistribuito, non è possibile aggiungere restrizioni per negare ad altre persone le libertà principali. Questa regola non entra in conflitto con le libertà principali, anzi le protegge.

%**************************************************************
\section{Licenze per il software libero}
Una licenza di software libero è una licenza libera, un testo legale caratterizzato da un aspetto contrattuale o para-contrattuale, che si applica ad un software per garantirne la libertà d'utilizzo, di studio, di modifica e di condivisione, ovvero per renderlo software libero. \\
La nascita del concetto di licenza applicata ad un software per renderlo libero combacia in parte con la nascita di \textit{GNU}, il primo sistema operativo completamente libero ideato da Richard Stallman nel 1983. Tutt'oggi il progetto GNU e la Free Software Foundation patrocinano attivamente il software distribuito sotto licenze libere e, in generale, la libertà digitale degli utenti. \\ \\
Di seguito vengono descritte alcune tra le licenze per il software libero utilizzate dalle librerie esterne necessarie allo sviluppo del software \textit{Biztortion}.

\subsection{GPL}
\label{sec:gpl}
La \textit{GNU General Public License}\footcite{site:gpl} (comunemente indicata con l'acronimo GNU GPL o semplicemente GPL) è una licenza \textit{fortemente copyleft} per software libero, originariamente stesa nel 1989 da Richard Stallman per patrocinare i programmi creati per il sistema operativo GNU. \\
La Free Software Foundation (FSF) detiene i diritti di copyright sul testo della GNU GPL, ma non detiene alcun diritto sul software da essa coperto. La GNU GPL non è liberamente modificabile: solo la copia e la distribuzione sono permesse. Per questo motivo solo la FSF può pubblicare nuove revisioni o versioni, l'ultima delle quali fu pubblicata il 29 giugno del 2007 sotto il nome di \textbf{GNU GPL v3}.

\subsection{MIT}
La \textit{Licenza MIT}\footcite{site:mit} è una licenza di software libero creata dal Massachusetts Institute of Technology (MIT). 
La licenza in questione è \textit{senza copyleft} e risulta molto permissiva, in quanto a differenza delle licenze software copyleft, la licenza MIT consente anche il riutilizzo all'interno di software proprietario, a condizione che tutte le copie del software o delle sue parti sostanziali includano una copia dei termini della licenza MIT e anche un avviso di copyright. La licenza MIT È anche una licenza \textit{GPL-compatibile}, cioè la GPL permette di combinare e ridistribuire tale software con altro che usa la licenza MIT.

\subsection{BSD}
Le \textit{licenze BSD} sono una famiglia di licenze permissive, \textit{senza copyleft}, per software libero. Il loro nome deriva dal fatto che la licenza BSD originale (detta anche licenza BSD con 4 clausole) fu usata originariamente per distribuire il sistema operativo Unix Berkeley Software Distribution (BSD), una revisione libera di UNIX sviluppata presso l'Università di Berkeley. \\
La versione originale è stata successivamente rivista e le sue discendenti sono più propriamente definite licenze BSD modificate. Due varianti della licenza, la Nuova Licenza BSD (o Licenza BSD Modificata)\footcite{site:bsd} e la Licenza semplificata BSD (o FreeBSD), sono state verificate come licenze di software libero \textit{GPL-compatible} dalla Free Software Foundation.

%**************************************************************
\section{Librerie utilizzate}
Per la realizzazione del software \textit{Biztortion} sono state utilizzate delle librerie sviluppate da terzi in modo da semplificare lo sviluppo di alcuni componenti. Di seguito vengono elencate e descritte tutte le librerie in questione, le quali utilizzano tutte delle licenze per software libero che sono \textit{GPL-compatible} rendendo più semplice la scelta della licenza di rilascio e distribuzione del software sopraccitato:
\begin{itemize}
    \item \textbf{JUCE}: il framework C++ in questione, necessario per la creazione dell'applicazione audio, utilizza termini di licenza a vari livelli, con termini diversi per ogni licenza disponibile: \textit{JUCE Personal} (per sviluppatori o start-up con entrate inferiori al limite di entrate di 50K USD; gratuito), \textit{JUCE Indie} (per le piccole imprese con limite di entrate inferiore a 500K USD; \$ 40/mese), \textit{JUCE Pro} (nessun limite di entrate; \$ 130/mese) e \textit{JUCE Educational} (nessun limite di entrate; gratuito per istituzioni educative in buona fede). Risulta infine possibile rilasciare un'applicazione sviluppata con JUCE sotto la \textbf{GNU GPLv.3}, rendendola software libero in modo che ne possa beneficiare tutta la comunità;
    \item \textbf{dRowAudio}: consiste in un modulo JUCE di terze parti progettato per lo sviluppo rapido di applicazioni audio; la libreria, sviluppata da \textit{David Rowland}, contiene diverse classi per l'elaborazione audio e vari elementi utili per la creazione dell'interfaccia grafica. Il modulo in questione è distribuito secondo i termini della \textbf{Licenza MIT} per il software libero;
    \item \textbf{ff\_meters}: consiste in un modulo JUCE di terze parti che offre un componente facile da usare per visualizzare una lettura di livello del segnale utilizzando un \textit{juce::AudioBuffer} in ingresso. La libreria, sviluppata da \textit{Daniel Walz} (Foleys Finest Audio Ltd.), deve essere utilizzata con il framework JUCE. Il modulo in questione è distribuito secondo i termini della \textbf{Licenza BSD a 3 clausole} (conosciuta anche come "Nuova" o "Modificata").
\end{itemize}

%**************************************************************
\section{Software Biztortion}
Il software \textit{Biztortion}, poiché è stato sviluppato con le librerie citate nella sezione precedente che utilizzano licenze per il software libero \textit{GPL-compatible}, viene distribuito secondo i termini stabiliti dalla licenza \hyperref[sec:gpl]{GNU GPLv.3}, ovvero la più recente delle GPL. In sostanza la licenza in questione, oltre ad affermare le quattro libertà fondamentali del software libero, garantisce che tutte le versioni migliorate a partire da quella originale che saranno distribuite dovranno essere libere a loro volta. \\ \\
Il software in questione viene rilasciato, insieme al presente documento per la discussione della tesi di laurea, alla versione \textit{1.0} ed è possibile trovare tutti i file corrispondenti al prodotto software in \href{https://github.com/killbizz/Biztortion}{questo repository remoto}.